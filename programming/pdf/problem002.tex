\documentclass[12pt]{jarticle}

\renewcommand{\thefootnote}{\fnsymbol{footnote}}

\title{群論演習}
\date{}

\begin{document}
\maketitle

\section*{問題001}
次の行列$A$について考える.
\[
  A = \left(
    \begin{array}{ccc}
      2 & 0 \\
      0 & 3 \\
    \end{array}
  \right)
\]


%(1), (2), (3),...
\begin{enumerate}
\renewcommand{\labelenumi}{(\arabic{enumi})}
\item $A^2$,$A^3$,$A^4$を計算しなさい.
\item 「推論」という言葉の意味を調べなさい.また,「帰納」と「演繹」について,その言葉の意味を例を用いながら説明しなさい.
\item $n$を自然数とする,$A^{n}$が以下のようになることを,数学的帰納法で示しなさい.
\[
  A^{n} = \left(
    \begin{array}{ccc}
      2^n & 0 \\
      0 & 3^n \\
    \end{array}
  \right)
\]
\item 数学的帰納法は「演繹」か「帰納」か.
\end{enumerate}


\footnote[0]{
作問:@math153arclight\\
  質問等はmino2357あっとgmail.comまで
}

\thispagestyle{empty}


\end{document}