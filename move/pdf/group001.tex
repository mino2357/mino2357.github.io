\documentclass[dvipdfmx,12pt]{jsarticle}
\usepackage{amsmath}
\usepackage{tikz}
\usetikzlibrary{cd}
%\renewcommand{\thefootnote}{\fnsymbol{footnote}}
\usepackage{url}
\title{無題--群論--001}
\author{itmz153}
\date{\today}

\begin{document}
\maketitle
\section*{とある会話}
K氏「任意のアーベル群$G(S, **)$に対して$a ** b = abc (c∈S)$としたとき$(S, **)$も群になるやんけ!\footnote{\url{https://twitter.com/****************************}許可待ち}」 \\

M氏「証明を\footnote{\url{https://twitter.com/************************************}許可待ち}」 \\

$\vdots$
\section*{群構造}
\subsection*{群の定義}
群とは,集合と\underline{適切}な演算の対である.すなわち集合を$G$,演算を$\phi$とすれば,$(G,\phi)$であり,演算について次の公理$\gamma_1 , \gamma_2$を満たせば$(G,\phi)$を群という.演算は2変数の写像,$G \times G \to G$である.
\begin{align*}
\gamma_1\ &:\ \forall x,y,z \in G[\phi(\phi(x,y),z) = \phi(x,\phi(y,z))] \\
\gamma_2\ &: \ \exists e \in G [\forall g \in G[\phi(g,e) = \phi(e,g) = g] \land \forall g \in G , \exists u \in G[\phi(g,u) = \phi(u,g) = e]]
\end{align*}
$\gamma_1$は演算の結合性,$\gamma_2$は,単位元の存在と逆元の存在性を述べている.今後,単位元は$e$と書き,$x \in G$の逆元は$g^{-1}$と書くことにする.
今後毎回$\phi$を書くとややこしいので,代わりに括弧を省き中置演算子$*$を使うことにする.
また,群$(G,*)$のことを,ただ単に群$G$と書くことがある.その場合,先に述べた適切な演算が$G$に入っていると思えば良い.
\subsection*{とある会話について}
$(G,*)$を群とする.ここで,$c \in G$を固定して,
\begin{align*}
Gc = \{g*c \mid g \in G\}
\end{align*}
なる集合を考える.$G$は群なので$Gc$も群である(表示は違うが同じ演算が自然に入ると思えば良い).ここで,群$Gc$に次のような演算$\langle \_, \_ \rangle_c$を導入する.
%
\begin{center}
\begin{tikzcd}[row sep=0.2cm]
\langle \_, \_ \rangle_c \ :\ Gc \times Gc\ar[r] & Gc \\
\langle x , y\rangle_c \ar[r,mapsto] &  (x * y) * c
\end{tikzcd}
\end{center}
%
実は,$Gc = G$(集合としても(元の演算を考えれば)群として等しい.)なので,次を考えることと等価である.
%
\begin{center}
\begin{tikzcd}[row sep=0.2cm]
\langle \_, \_ \rangle_c \ :\ G \times G\ar[r] & G \\
\langle x , y\rangle_c \ar[r,mapsto] &  x * y * c
\end{tikzcd}
\end{center}
%
この演算が,群の演算の公理$\gamma_1, \gamma_2$を満たしているか確認していこう.
演算が閉じていることは明らか.はじめに,結合法則がなりたつか見ていこう.
\begin{align*}
\langle \langle x, y \rangle_c , z \rangle_c &= \langle (x*y)*c , z\rangle_c \\
&= x*y*c*z*c \\
&= \cdots
\end{align*}
なんだかこのままだと何もできそうにないので,\underline{$c$を群の中心$Z(G)$から取ってこよう.}
群の中心とは,$\{ x \in G \mid \forall g [g*x = x*g] \}$のことである.群の中心は$G$の正規部分群となるなど面白い性質がある(確認せよ).

\underline{今後,$c \in Z(G)$とする.}
\begin{align*}
\langle \langle x, y \rangle_c , z \rangle_c &= x*y*c*z*c \\
&= x * y*z*c *c \\
&= x * \langle y,z \rangle * c \\
&= \langle x, \langle y,z \rangle_c \rangle_c
\end{align*}
無事に新しく導入した演算は結合法則を満たすことがわかった.
次に,単位元の存在性を確認しよう.
\begin{align*}
\exists e ,\forall g [\langle g,e , \rangle_c = \langle e,g \rangle_c = g]
\end{align*}
いま,単位元の候補として$e':=c^{-1}$と置こう.任意の$g$に対して,
\begin{align*}
\langle g,e' \rangle_c &= \langle g,c^{-1} \rangle_c \\
&=g * c^{-1} * c = g
\end{align*}
\begin{align*}
\langle e',g \rangle_c &= \langle c^{-1},g \rangle_c \\
&=c^{-1} * g * c = g
\end{align*}
%故に演算$\langle,\rangle$によって定義される群には,単位元が存在する.(一意性は各自確認せよ.)
次に逆元について考察する.
\begin{align*}
\forall g,\exists u [\langle g,u \rangle_c = \langle u,g \rangle_c = e' = c^{-1}]
\end{align*}
$u:=g^{-1}*c^{-2}$とすれば,
\begin{align*}
\langle g,u \rangle_c &= \langle g,g^{-1}*c^{-2} \rangle_c \\
&=g * g^{-1}*c^{-2} *c \\
& = c^{-1} = e'
\end{align*}
\begin{align*}
\langle u,g \rangle_c &= \langle g^{-1}*c^{-2},g \rangle_c \\
&=g^{-1}*c^{-2} * g *c\\
& = c^{-1} = e'
\end{align*}
よって,$e':=c^{-1}は単位元の公理を満たす.$故に,新たな群の演算$\langle , \rangle_c$は演算の公理$\gamma_1, \gamma_2$を満たしているので,$c \in Z(G)$なら$(G,\langle , \rangle_c)$は群である.

さて,この群はなんであろうか?

%\footnote[0]{質問等はmino2357あっとgmail.comまで}

\thispagestyle{empty}


\end{document}