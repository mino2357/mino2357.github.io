\documentclass[dvipdfmx,12pt]{jsarticle}
\usepackage{amsmath}
\usepackage[usenames]{color} %フォントカラー
\usepackage{amssymb} %数式記号
\usepackage[utf8]{inputenc} %発音区別符アルファベットの直接入力
\usepackage{amsthm}
\usepackage{amscd}
\usepackage{pb-diagram}
\usepackage{ulem}
%\usepackage[all]{xy}
\usepackage{tikz}
\usetikzlibrary{cd}
%\usepackage{mathptmx}
\renewcommand{\thefootnote}{\fnsymbol{footnote}}

\theoremstyle{definition}
\newtheorem{theorem}{命題}
\newtheorem*{theorem*}{命題}
\newtheorem{definition}[theorem]{定義}
\newtheorem*{definition*}{定義}
\renewcommand\proofname{{\bf{証明}}}

%\title{数学とプログラミング演習}
\date{}

\begin{document}
%\maketitle

\subsection*{距離空間}
自分のための定義,定理,証明まとめ.
%
\begin{definition}[収束]
距離空間$(X,d)$において,\underline{点列$\{ a_{i} \}_{i \in \mathbb{N}}$が$a\in X$に収束する}とは,\\
$
\ \ \ \ \ \ \ \ \ \ \ \ \forall \varepsilon >0,\ \exists N \in \mathbb{N},\ \forall n > N,\ d(a_{n}, a) < \varepsilon.
$
\end{definition}
%
\begin{definition}[収束列]
距離空間$(X,d)$において,\underline{点列$\{ a_{i} \}_{i \in \mathbb{N}}$が収束列}であるとは,\\
$
\ \ \ \ \ \ \ \ \ \ \ \ \exists a \in X,\ \forall \varepsilon > 0,\ \exists N \in \mathbb{N},\ \forall n>N,\ d(a_{n},a) < \varepsilon.
$
\end{definition}
%
\begin{definition}[Cauchy列]
距離空間$(X,d)$において,\underline{点列$\{ a_{i} \}_{i \in \mathbb{N}}$がCauchy列}であるとは,\\
$
\ \ \ \ \ \ \ \ \ \ \ \ \forall \varepsilon > 0,\ \exists N \in \mathbb{N},\ \forall m,\forall n > N,\ d(a_{m}, a_{n}) < \varepsilon.
$
\end{definition}
%
\begin{definition}[点列コンパクト]
距離空間$(X,d)$が\underline{点列コンパクト}であるとは,任意の点列が収束部分列を持つ空間のことである.
\end{definition}
%
\begin{definition}[完備]
距離空間$(X,d)$が\underline{完備}であるとは,$X$における任意のCauchy列が収束することである.
\end{definition}
%
\begin{definition}[全有界]
距離空間$(X,d)$が\underline{全有界}であるとは,任意の$\varepsilon > 0$に対して,$X$の有限個の点,$x_{1}, x_{2}, x_{3}, \ldots , x_{n}$を選んで,\\
$$
X = \bigcup_{k=1}^{n} N(x_{1}; \varepsilon)
$$
となるようにできることである.ただし,$N(a,\varepsilon)$は,点$a$の$\varepsilon$近傍,すなわち
$
N(a;\varepsilon) := \{ x \in X \mid d(a,x) < \varepsilon \}のことである.
$
\end{definition}
%
\begin{theorem}
距離空間$(X,d)$が \\
$$
「点列コンパクト」 \Longrightarrow 「全有界」かつ「完備」
$$
であることを示せ.
\end{theorem}
%
\begin{proof}
$X$から,点列$\{ a_{i} \}_{i \in \mathbb{N}}$を取ってくる.今,$X$は点列コンパクト空間なので,以下のような,部分列$\{ a_{k(i)} \}_{i \in \mathbb{N}}$を収束させるような狭義単調増加函数
$$
k: \mathbb{N} \to \mathbb{N}
$$
が存在する.
\end{proof}

\subsection*{位相空間論についてのある問題}
以下,いろいろ問題がありすぎる.

\begin{definition}
$\forall U \in \mathcal{O}$になんらかの正の実数$\mathrm{size}(U) \in \mathbb{R}_{\ge 0}$($U$の\underline{サイズ}と呼ぶことにする.)が付与された位相空間$(X, \mathcal{O}, \mathrm{size})$が\underline{全有界}であるとは,任意の$E>0$に対して,$n \in \mathbb{N}$が存在して,それぞれのサイズが$E$以下であるような$U_{1}, U_{2}, \ldots ,U_{n}$が存在し,その合併がXの(有限)開被覆になっていることである.
\end{definition}

↑これもどうかと思うよ.

\begin{definition}
位相空間$(X, \mathcal{O})$について,点$p \in X$の\underline{近傍}とは,点$p$を含む$X$の開集合$U$を含む部分集合$V$のことをいう.
\end{definition}

\begin{definition}
位相空間$(X,\mathcal{O})$において,\underline{点列$\{ a_{i} \}_{i \in \mathbb{N}}$が点$a$に収束する}と,$a$の任意の近傍$U$に対して,$\exists N \in \mathbb{N}$,$\forall n \ge N \Longrightarrow a \in U$が成り立つことである.
\end{definition}

\begin{definition}
位相空間$(X,\mathcal{O})$が\underline{点列コンパクト}であるとは,任意の点列が収束部分列を持つことである.
\end{definition}

\begin{definition}
\sout{位相空間$(X,\mathcal{O})$が\underline{完備}であるとは,$\forall i\in \mathbb{N},\ a_{i} \in X$なる任意の点列$\{ a_{i} \}_{i \in \mathbb{N}}$が}\\
\sout{収束することである.}
\end{definition}

この設定で以下を証明しないさい.逆も成り立つ場合,それを証明せよ.

\begin{theorem*}
位相空間$(X, \mathcal{O})$について, \\
\ \ \ \ \ \ \ \ \ 「点列コンパクト」 $\Longrightarrow$ 「全有界」かつ「完備」
\end{theorem*}



\footnote[0]{
質問等はmino2357あっとgmail.comまで
}

\thispagestyle{empty}


\end{document}