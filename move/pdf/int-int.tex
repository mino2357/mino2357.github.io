\documentclass[12pt]{jarticle}

\usepackage{amsmath,amssymb}
\renewcommand{\thefootnote}{\fnsymbol{footnote}}

\date{}

\begin{document}

\section*{重積分の問題について}
次の重積分を求めるため,累次積分の形にして計算したい.\footnote{累次積分で求められることは省略.証明は三宅著「入門微分積分」など参照.}.
\begin{align*}
\iint_{D} x y^{2}  dxdy, \ \ D=\{(x, y) \mid x \le 1,\ y \le 1,\ y \ge -x + 1 \}
\end{align*}
%
\subsection*{$x$から累次積分する}
領域$D$についての条件を変形すると,($x$の範囲が$y$に依存した形)
\begin{align*}
\left\{
	\begin{array}{ll}
	x \in [1-y, \infty) \cap (-\infty, 1] \\
	y \in (-\infty, 1]
	\end{array}
\right.
\end{align*}
$x$の区間については次のように変形できて,
\begin{align*}
\left\{
	\begin{array}{ll}
	x \in [1-y, 1] \\
	y \in (-\infty, 1]
	\end{array}
\right.
\end{align*}
また,ここで,
\begin{align*}
x \in [1-y,1] \neq \emptyset
\end{align*}
なので,
\begin{align*}
1-y \le 1
\end{align*}
解くと,$y$が取らなければいけない範囲は,
\begin{align*}
0 \le y
\end{align*}
故に,
\begin{align*}
\left\{
	\begin{array}{ll}
	x \in [1-y, 1] \\
	y \in [0, 1]
	\end{array}
\right.
\end{align*}
以上から,$D$について,
\begin{align*}
D &= \{(x,y) \mid y \in [-\infty, 1]. x \in [1-y,1] \} \\
&= \{(x,y) \mid y \in [0, 1]. x \in [1-y,1] \}
\end{align*}
となる.
よって,累次積分は,
\begin{align*}
\int_{0}^{1} \left( \int_{1-y}^{1} x y^{2} dx \right) dy
\end{align*}
を計算すれば良い.
%
%
%
%
\subsection{$y$から累次積分する}
領域$D$についての条件を変形すると,($y$の範囲が$x$に依存した形)
\begin{align*}
\left\{
	\begin{array}{ll}
	x \in ( \infty, 1] \\
	y \in [1-x, 1]
	\end{array}
\right.
\end{align*}
ここで,
\begin{align*}
y \in [1-x,1] \neq \emptyset
\end{align*}
なので,
\begin{align*}
\left\{
	\begin{array}{ll}
	x \in [0, 1] \\
	y \in [1-x, 1]
	\end{array}
\right.
\end{align*}
ゆえに領域$D$は,
\begin{align*}
D = \{(x,y) \mid x \in [0,1],\ y \in [1-x,1] \} 
\end{align*}
よって,累次積分は,
\begin{align*}
\int_{0}^{1} \left( \int_{1-x}^{1} x y^{2} dy \right) dx
\end{align*}
となり,これを計算すれば良い.
\footnote[0]{
質問等はmino2357あっとgmail.comまで
}

\thispagestyle{empty}


\end{document}