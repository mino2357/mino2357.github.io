\documentclass[12pt]{jarticle}
\usepackage{amsmath}

\renewcommand{\thefootnote}{\fnsymbol{footnote}}

%\title{数学とプログラミング演習}
\date{}

\begin{document}
%\maketitle

\section*{常微分方程式の数値解法入門001}
ロケットの打ち上げを考える.ロケットの質量は$m$とする.直交座標($x$軸は右方向,$y$軸は上方向に取る)における原点$O$からロケットを発射するとする.時刻$t$でのロケットの位置を$\vec{r} = (x(t), y(t))$とする.はじめの速さは$v$(固定)で打ち上げる.初期条件は以下の通り,
%
\begin{align*}
\dfrac{dx(0)}{dt} = v \cos \theta \\
\dfrac{dy(0)}{dt} = v \sin \theta
\end{align*}
%
この設定のもと,ITMZは$\theta$をどのように変化させると,どのくらい距離を飛ばせるかを考えている.
%
\subsection*{問題001}
重力加速度を$g$として,ITMZは仮定を簡略化したロケットモデルとして以下の運動方程式を立てた.
$x$軸方向は,
\begin{align*}
m \dfrac{d^2 x(t)}{dt^2} = 0
\end{align*}
$y$軸方向は,
\begin{align*}
m \dfrac{d^2 y(t)}{dt^2} = -mg
\end{align*}
となった.

この時,最大飛距離を求めなさい.また,その時の,発射角度$\theta$も求めなさい.
%
\subsection*{問題002}
ITMZは実験との比較により,上記のロケットモデルでは不十分なことがわかり,モデル方程式を修正することにした.ここで,速度に比例する空気抵抗を考慮にいれることにした.運動方程式は以下のようになった.(確認せよ.)
$x$軸方向は,
\begin{align*}
m \dfrac{d^2 x(t)}{dt^2} = - k \dfrac{dx(t)}{dt}
\end{align*}
$y$軸方向は,
\begin{align*}
m \dfrac{d^2 y(t)}{dt^2} = - k \dfrac{dy(t)}{dt} -mg
\end{align*}
ここで,$k$は空気抵抗の比例係数である.

この時,最大飛距離を求めなさい.また,その時の,発射角度$\theta$も求めなさい.
%
\subsection*{問題003}
ITMZは実験との比較により,上記のロケットモデルではまだ不十分なことがわかり,モデル方程式を修正することにした.ここで,速度の2乗に比例する空気抵抗も考慮にいれることにした.運動方程式は以下のようになった.(確認せよ.)
$x$軸方向は,
\begin{align*}
m \dfrac{d^2 x(t)}{dt^2} = - k \dfrac{dx(t)}{dt} - l \dfrac{dx(t)}{dt} \sqrt{ \left( \dfrac{dx(t)}{dt} \right)^2 + \left( \dfrac{dy(t)}{dt} \right)^2 } 
\end{align*}
$y$軸方向は,
\begin{align*}
m \dfrac{d^2 y(t)}{dt^2} = - k \dfrac{dy(t)}{dt} -  l \dfrac{dy(t)}{dt} \sqrt{ \left( \dfrac{dx(t)}{dt} \right)^2 + \left( \dfrac{dy(t)}{dt} \right)^2 } - mg
\end{align*}
ここで,$l$は速度の2乗に比例する空気抵抗の比例係数である.

この時,最大飛距離を(数値計算で)求めなさい(パラメータ$k, l$は勝手に与えて良い.).また,ロケットの軌跡もグラフにプロットせよ.この問題については,コンピュータを用いて計算しても良い(推奨).
%
\subsection*{問題004(ここからは興味がある人のみが解けば良い.)}
ITMZは実験との比較により,上記のロケットモデルではまだ不十分なことがわかり,モデル方程式を修正することにした.ここで,燃料の消費によるロケットの質量変化も考慮にいれることにした.

この時,上記で行ってきたものと同様の考察をせよ.
%
\subsection*{問題005}
ITMZは実験との比較により,上記のロケットモデルではまだ不十分なことがわかり,モデル方程式を修正することにした.ITMZは高度によって重力定数が変わることに気づいた.そのもとで,上記ロケットモデルにこの効果を入れた数理モデルを考えよ.

この時,上記で行ってきたものと同様の考察をせよ.


\footnote[0]{
質問等はmino2357あっとgmail.comまで
}

\thispagestyle{empty}


\end{document}