\documentclass[12pt]{jarticle}
\usepackage{amsmath}

\renewcommand{\thefootnote}{\fnsymbol{footnote}}

%\title{数学とプログラミング演習}
\date{}

\begin{document}
%\maketitle

\section*{常微分方程式の数値解法入門002}
以下の微分方程式を数値的に解きたい.ただし$y$は$x$の関数とする.
\begin{align*}
\dfrac{dy}{dx} = -y
\end{align*}
初期条件は,
\begin{align*}
\dfrac{dy(0)}{dx} = a
\end{align*}
とする.
\subsection*{問題001}
上記,常微分方程式の初期値問題の厳密解を求めなさい.
%
\subsection*{問題002}
オイラー法によって,数値計算をし,解を図示しなさい.また,厳密解と比較せよ.
\subsection*{問題003}
ルンゲ・クッタ法によって数値解を求め,厳密解と比較せよ.

\footnote[0]{
質問等はmino2357あっとgmail.comまで
}

\thispagestyle{empty}


\end{document}