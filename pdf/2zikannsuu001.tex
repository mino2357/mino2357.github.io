\documentclass[12pt]{jarticle}

\usepackage{amsmath,amssymb}
\usepackage{cases}
\pagestyle{empty}

\renewcommand{\thefootnote}{\fnsymbol{footnote}}

\date{}

\begin{document}

\section*{問題4}
以下の放物線について考える.
\begin{align}
\label{001}
y = 2x^2 -8x -6
\end{align}
まず(\ref{001})を式変形し,頂点$A$の座標を求める
\begin{align*}
y &= 2x^2 - 8x -6 \\
&= 2 (x^2 - 4x) - 6 \\
&= 2 \{(x-2)^2-4\} - 6 \\
&= 2 (x-2)^2 - 8 - 6 \\
&= 2 (x-2)^2 -14
\end{align*}
よって,頂点の座標$A$は,
\begin{align*}
A(2,-14)
\end{align*}
となる.これが1番の答えとなる.

次に$A$を頂点として点$(6,-10)$を通る2次関数を求める.$A$を頂点とするので,ある定数$a$を用いて,求めるべき2次関数は次のように書ける.
\begin{align}
\label{002}
y = a(x-2)^2 - 14
\end{align}
これが,点$(6,-10)$を通るので,(\ref{002})式に代入すると,
\begin{align*}
-10 &= a(6-2)^2 -14 \\
-10 &=16a -14 \\
16a &= 4 \\
a&= \dfrac{1}{4}
\end{align*}
よって,求めるべき2次関数は,
\begin{align*}
y = \dfrac{1}{4} (x-2)^2 -14
\end{align*}
となる.これが2番の答えである.
%

\section*{問題5}
まず,2次関数の軸が$x=3$なので,求める関数は,ある定数$a, b$を用いて
\begin{align}
\label{003}
y = a(x-3)^2 + b
\end{align}
と書ける.ここで,2点$(-2, 0), (7, -18)$を通るので,(\ref{003})に代入して,
\begin{subnumcases}
  {}
0 = a(-2-3)^2 + b & \notag \\
-18 = a(7-3)+b & \notag
\end{subnumcases}
これらを整理すると,以下の連立1次方程式を得る.
\begin{subnumcases}
  {}
\label{004}
25a + b = 0 & \\
\label{005}
16a + b = -18 & 
\end{subnumcases}
式$(\ref{004})-(\ref{005})$を計算すれば,
\begin{align*}
18 = 9a
\end{align*}
これを解いて,
\begin{align*}
a = 2
\end{align*}
ここで出た,$a=2$を(\ref{004})に代入すれば,
\begin{align*}
b = -50
\end{align*}
を得る.よって,求めるべき2次関数は
\begin{align*}
y &= 2(x-3)^2-50 \\
  &=2x^2 -12x - 32
\end{align*}
である.これが1番の答えである.

次に2番を考える.2次関数は一般に定数$a, b, c$を用いて,
\begin{align*}
y = ax^2 + bx + c
\end{align*}
のように表せるので,これに3点の座標を代入して,未知定数$a, b, c$を求めることにする.
\begin{subnumcases}
  {}
\label{006}
-15 = 4a - 2b +c & \\
\label{007}
9 = a + b +c  &\\
\label{008}
15 = 9a + 3b +c &
\end{subnumcases}
式$(\ref{006})-(\ref{007})$を計算すれば,
\begin{align}
\label{009}
a-b = -8
\end{align}
式$(\ref{007})-(\ref{008})$を計算すれば,
\begin{align}
\label{010}
4a+b = 3
\end{align}
式$(\ref{009})+(\ref{010})$を計算すれば,
\begin{align*}
a = -1
\end{align*}
(\ref{009})にこれを代入すれば,
\begin{align*}
b = 7
\end{align*}
今,得られた結果$a =-1, b = 7$を式(\ref{007})に代入して$c$を求めると,
\begin{align*}
c = 3
\end{align*}
以上をまとめて,求めるべき2次関数は,
\begin{align*}
y = -x^2 + 7x +3
\end{align*}
となる.これが2番の答えである.


答えが出たら必ず与えられた点がその曲線上にあることを確認すること.以上.

\footnote[0]{
質問等はmino2357あっとgmail.comまで(またはDMで)
}

\thispagestyle{empty}


\end{document}