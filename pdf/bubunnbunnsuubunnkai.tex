\documentclass[12pt]{jsarticle}
\usepackage{amsmath}
\usepackage{amsfonts}

\renewcommand{\thefootnote}{\fnsymbol{footnote}}

%\title{数学とプログラミング演習}
\date{}

\begin{document}
%\maketitle

\section*{部分分数分解と級数の極限}
高校数学.発展的話題まで.
\subsection*{一つの例}
まず,はじめに次の級数(数列の和のこと)を求める.
\begin{align*}
\sum_{k=1}^{n} \dfrac{1}{k(k+1)}
\end{align*}
これは,以下のように式変形を行えば,求めることが出来る.
\begin{align*}
\sum_{k=1}^{n} \dfrac{1}{k(k+1)} &= \sum_{k=1}^{n} \left( \dfrac{1}{k} - \dfrac{1}{k+1} \right) \\
&=1 - \dfrac{1}{n+1} \left( = \dfrac{n}{n+1} \right)
\end{align*}
このことから,次の極限値がわかる.
\begin{align*}
\lim_{n \to \infty} \sum_{k=1}^{n} \dfrac{1}{k(k+1)} = 1
\end{align*}
この極限値は以下のオイラーによって求められたバーゼル問題を考えると不思議なことに思われる.
\begin{align*}
\lim_{n \to \infty} \sum_{k=1}^{n} \dfrac{1}{k^2} = \dfrac{\pi^2}{6}
\end{align*}
\begin{flushright}
(1735{\mbox{年}}, Leonhard Euler)
\end{flushright}

\subsection*{一つの拡張}
上の問題を受けて,次の級数を求める.
\begin{align*}
\sum_{k=1}^{n} \dfrac{1}{k(k+1)(k+2)}
\end{align*}
まずは素直に求めてみよう.
\begin{align*}
\sum_{k=1}^{n} \dfrac{1}{k(k+1)(k+2)} =& \sum_{k=1}^{n} \dfrac{1}{k}\dfrac{1}{(k+1)(k+2)} \\
=&\sum_{k=1}^{n} \dfrac{1}{k} \left( \dfrac{1}{k+1} - \dfrac{1}{k+2} \right) \\
=&\sum_{k=1}^{n} \dfrac{1}{k(k+1)} - \sum_{k=1}^{n} \dfrac{1}{k(k+2)} \\
=&\left( 1- \dfrac{1}{n+1} \right) - \sum_{k=1}^{n} \dfrac{1}{2} \left( \dfrac{1}{k} - \dfrac{1}{k+2} \right) \\
=&\left( 1- \dfrac{1}{n+1} \right) - \dfrac{1}{2} \left\{ \left( \dfrac{1}{1} - \dfrac{1}{3} \right)  + \left( \dfrac{1}{2} - \dfrac{1}{4} \right) + \left( \dfrac{1}{3} - \dfrac{1}{5} \right) \right. \\
&\left.  + \cdots + \left( \dfrac{1}{n-2} - \dfrac{1}{n} \right) + \left( \dfrac{1}{n-1} - \dfrac{1}{n+1} \right) + \left( \dfrac{1}{n} - \dfrac{1}{n+2} \right) \right\} \\
=& \left( 1 - \dfrac{1}{n+1} \right) - \dfrac{1}{2} \left( \dfrac{1}{1} + \dfrac{1}{2} - \dfrac{1}{n+1} - \dfrac{1}{n+2} \right) \\
=& \left( 1 - \dfrac{1}{n+1} \right) - \left( \dfrac{3}{4} - \dfrac{1}{2(n+1)} - \dfrac{1}{2(n+2)} \right) \\
=& \dfrac{1}{4} - \dfrac{1}{2(n+1)} + \dfrac{1}{2(n+2)}  = \dfrac{1}{4} - \dfrac{1}{2(n+1)(n+2)}
\end{align*}
よって,この級数の極限値は,
\begin{align*}
\lim_{n \to \infty}\sum_{k=1}^{n} \dfrac{1}{k(k+1)(k+2)} = \dfrac{1}{4}
\end{align*}
である.

\subsection*{もう一つの計算}
次のように部分分数分解されるとしよう.ただし$A,B,C,D$は定数とする.
\begin{align*}
\dfrac{1}{k(k+1)(k+2)} = \dfrac{Ak+B}{k(k+1)} + \dfrac{Ck+D}{(k+1)(k+2)}
\end{align*}
これを$A,B,C,D$について上手く解けば,
\begin{align*}
\dfrac{1}{k(k+1)(k+2)} = \dfrac{1}{2} \left( \dfrac{1}{k(k+1)} + \dfrac{1}{(k+1)(k+2)} \right)
\end{align*}
一般の場合はこのようにすれば良いが,これは少し計算が煩雑である.工夫して計算しよう.
\begin{align*}
\dfrac{1}{k(k+1)(k+2)} =& \dfrac{1}{k+1} \left( \dfrac{1}{k(k+2)} \right) \\
=&\dfrac{1}{k+1} \left\{ \dfrac{1}{2} \left( \dfrac{1}{k} - \dfrac{1}{k+2} \right) \right\} \\
=&\dfrac{1}{2} \left( \dfrac{1}{k(k+1)} - \dfrac{1}{(k+1)(k+2)} \right)
\end{align*}
このように式変形すれば,直ちに導出できる.級数についても,
\begin{align*}
\sum_{k=1}^{n}\dfrac{1}{k(k+1)(k+2)} =& \dfrac{1}{2} \sum_{k=1}^{n} \left( \dfrac{1}{k(k+1)} - \dfrac{1}{(k+1)(k+2)} \right) \\
=& \dfrac{1}{2} \left\{ \left( \dfrac{1}{2} - \dfrac{1}{6} \right) + \left( \dfrac{1}{6} - \dfrac{1}{12} \right) + \cdots + \left( \dfrac{1}{n(n+1)} - \dfrac{1}{(n+1)(n+2)} \right) \right\} \\
=& \dfrac{1}{4} - \dfrac{1}{2(n+1)(n+2)}
\end{align*}
よって,同様に,
\begin{align*}
\lim_{n \to \infty}\sum_{k=1}^{n} \dfrac{1}{k(k+1)(k+2)} = \dfrac{1}{4}
\end{align*}

\subsubsection*{練習問題}
次の極限値を求めよ.
\begin{align*}
\lim_{n \to \infty}\sum_{k=1}^{n} \dfrac{1}{k(k+1)(k+2)(k+3)}
\end{align*}
\subsubsection*{研究}
\begin{align*}
\lim_{n \to \infty}\sum_{k=1}^{n} \dfrac{1}{k(k+1)\cdots(k+m-1)(k+m)}
\end{align*}

\subsection*{一つの一般化}
$k$と$k+1$の差は$1$であった.ここでは,一般化して,差を$\alpha(>0)$としよう.
\begin{align*}
\sum_{k=1}^{n} \dfrac{1}{k(k+\alpha)} \ \ \ \ \ (\alpha \neq -1)
\end{align*}
この級数と極限値を求めたい.その前に準備が必要である.まずは,
\begin{align*}
\sum_{k=1}^{n} \dfrac{1}{k\left(k+\dfrac{1}{2}\right)}
\end{align*}
を考えよう.

\subsection*{ガンマ関数}
実部が正となる複素数$z \in \mathbb{C}$について,次の積分で定義される関数
\begin{align*}
\Gamma (z)=\int _{0}^{\infty }t^{z-1}e^{-t}\,dt\qquad (\Re {z}>0)
\end{align*}
をガンマ関数と呼ぶ.次の式は階乗の一つの一般化として見ることができる.この等式の証明は部分積分を用いれば証明できる.
\begin{align*}
&\Gamma (z+1)=z\Gamma (z)\\
&\Gamma (1) = 1
\end{align*}
階乗と関連付けると,
\begin{align*}
\Gamma (n+1)=n!
\end{align*}
一般の複素数$z$に対しては,次の解析接続が知られている.
\begin{align*}
\Gamma (z)=\lim _{n\to \infty }{\dfrac {n^{z}n!}{\prod_{k=0}^{n}{(z+k)}}}
\end{align*}

\subsection*{ディガンマ関数}
ガンマ関数を用いて,次のディガンマ関数を定義する.
\begin{align*}
\psi (z)={\frac  {d}{dz}}\log {\Gamma (z)}={\frac  {\Gamma '(z)}{\Gamma (z)}}
\end{align*}
ここで,ガンマ関数を微分することで,
\begin{align*}
\Gamma'(z+1) = z\Gamma'(z) + \Gamma(z)
\end{align*}
ここで,両辺を$\Gamma(z+1)$で割ると,
\begin{align*}
\dfrac{\Gamma'(z+1)}{\Gamma(z+1)} &= \dfrac{z\Gamma'(z) + \Gamma(z)}{\Gamma(z+1)} \\
&= \dfrac{z\Gamma'(z) + \Gamma(z)}{z\Gamma(z)} \\
&= \dfrac{\Gamma'(z)}{\Gamma(z)} + \dfrac{1}{z}
\end{align*}
よって,ディガンマ関数を用いると,
\begin{align*}
\psi(z+1) = \psi(z) + \dfrac{1}{z}
\end{align*}
少しの計算で次の等式を得る.$H_n$はハーモニックナンバーである.(証明は読者に任せる.)今,$n \in \mathbb{N}$としよう.
\begin{align*}
\psi (n)= & \psi(1) +\sum _{{k=1}}^{{n-1}}{\frac  {1}{k}}\\
= & -\gamma +H_{{n-1}}
\end{align*}
また,半整数について,
\begin{align*}
\psi (n+1/2) =& \psi\left( 1/ 2 \right)+\sum _{{k=0}}^{{n-1}}{\frac  {1}{k+\dfrac{1}{2}}} \\
=& -\gamma -2\log {2}+\sum _{{k=0}}^{{n-1}}{\frac  {1}{k+\dfrac{1}{2}}} \\
=& -\gamma -2\log {2}+ 2 + \sum _{{k=1}}^{{n-1}}{\frac  {1}{k+\dfrac{1}{2}}}
\end{align*}
ここで,
\begin{align*}
\psi(1) &= -\gamma \\
\psi\left( \dfrac{1}{2} \right) &= -\gamma -2\log 2
\end{align*}
を用いた.(証明略.)
\begin{align*}
\sum_{k=1}^{n-1}\dfrac{1}{k\left(k+\dfrac{1}{2}\right)} =& 2 \sum_{k=1}^{n-1} \left( \dfrac{1}{k} - \dfrac{1}{k+\dfrac{1}{2}} \right) \\
=& 2 \left( \sum_{k=1}^{n-1} \dfrac{1}{k} - \sum_{k=1}^{n-1} \dfrac{1}{k+\dfrac{1}{2}} \right) \\
=& 2 \left\{ \psi(n)+ \gamma - \left( \psi\left( n + 1/2 \right) + \gamma + 2\log 2 - 2 \right) \right\} \\
=& 2 \left( \psi(n) - \psi\left( n + 1/2 \right) + 2 - 2\log 2 \right)
\end{align*}
ここで,ディガンマ関数の漸近展開は,
\begin{align*}
\psi(n) =& \log z - \dfrac{1}{2z} - \dfrac{1}{12z^2} + \dfrac{1}{120z^4} + \mathcal{O}\left( \left( \dfrac{1}{z} \right)^6 \right) \\
\psi(n+1/2) =& \log z + \dfrac{1}{24z^2} - \dfrac{7}{960z^4} + \mathcal{O}\left( \left( \dfrac{1}{z} \right)^6 \right)
\end{align*}
よって,これらの差は漸近展開は漸近冪級数展開となる.
\begin{align*}
\psi(n) - \psi(n+1/2) = - \dfrac{1}{2z} - \dfrac{1}{8z^2} + \dfrac{1}{64z^4} + \mathcal{O}\left( \left( \dfrac{1}{z} \right)^5 \right)
\end{align*}
ゆえに,
\begin{align*}
\sum_{k=1}^{n-1}\dfrac{1}{k\left(k+\dfrac{1}{2}\right)} =& 2 \left( \psi(n) - \psi\left( n + 1/2 \right) + 2 - 2\log 2 \right)
\end{align*}
に対して,$n \to \infty$の極限を取れば,
\begin{align*}
\sum_{k=1}^{\infty}\dfrac{1}{k\left(k+\dfrac{1}{2}\right)} =& 2 \left( 2 - 2\log 2 \right) \\
=& 4(1 - \log 2)
\end{align*}
となる\footnote{$\alpha=1/2$の時はさらに初等的な別証明がある.}.これを参考に,
\begin{align*}
\sum_{k=1}^{n} \dfrac{1}{k(k+\alpha)} \ \ \ \ \ (\alpha \neq -1)
\end{align*}
を求めることは読者に任せる.
\subsection*{その他の話題}
$\alpha \to 0$を考えると,
\begin{align*}
\lim_{x \to 0}\dfrac{H_{x}}{x} = \dfrac{\pi^2}{6}
\end{align*}
また,計算を進めると,
\begin{align*}
\dfrac{1}{2}\sum_{k=1}^{\infty}\dfrac{H_{k}}{k^2} = \zeta(3)
\end{align*}
別表現は,
\begin{align*}
\dfrac{1}{2}\sum_{k=1}^{\infty} \left\{ \dfrac{1}{k^2}\left( \dfrac{\Gamma'(k+1)}{\Gamma(k+1)} + \gamma \right) \right\} = \zeta(3)
\end{align*}
となる.これを一般化して数式処理ソフトにより計算させると,
\begin{align*}
\sum_{k=1}^{\infty}\dfrac{H_{k}}{k^{15}}=-\zeta (7) \zeta (9)-\zeta (5) \zeta (11)-\zeta (3) \zeta(13)+\frac{3617 \pi ^{16}}{76621545000}
\end{align*}
や,
\begin{align*}
\sum_{k=1}^{\infty}\dfrac{H_{k}}{k^{21}}=-\frac{\zeta (11)^2}{2}-\zeta (9) \zeta (13)-\zeta (7) \zeta (15)-\zeta (5) \zeta(17)-\zeta (3) \zeta (19)+\frac{77683 \pi ^{22}}{1169378864403750}
\end{align*}
\begin{align*}
\sum_{k=1}^{\infty}\dfrac{H_{k}}{k^{23}}=&-\zeta (11) \zeta (13)-\zeta (9) \zeta (15)-\zeta (7) \zeta (17)-\zeta (5) \zeta(19)-\zeta (3) \zeta (21) \\
&+\frac{236364091 \pi ^{24}}{32307131514201043500}
\end{align*}
を得る.

\footnote[0]{
質問等はmino2357あっとgmail.comまで
}

\thispagestyle{empty}


\end{document}