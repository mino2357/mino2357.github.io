\documentclass[12pt]{jarticle}
\usepackage{amsmath}

\renewcommand{\thefootnote}{\fnsymbol{footnote}}

%\title{数学とプログラミング演習}
\date{}

\begin{document}
%\maketitle

\section*{積分と極限についての問題}

次の自然数$n$に添字付けられた関数$f_n$について考える.
\begin{align*}
  f_{n}(x) =
\left\{ \begin{array}{ll}
4n^{2}x & \left(0 \le x < \dfrac{1}{2n}\right) \\
-4n^{2}x + 4n & \left(\dfrac{1}{2n} \le x < \dfrac{1}{n}\right) \\
0 & \left(\rm{otherwise}\right) \\
\end{array} \right.
\end{align*}
(1)次の積分の値を求めよ.
\begin{align*}
\int_{0}^{1} f_{1}(x) dx
\end{align*}
(2) 次の積分の値を求めよ.
\begin{align*}
\int_{0}^{1} f_{3}(x) dx
\end{align*}
(3) 次の極限の値を求めよ.
\begin{align*}
\lim_{n \to \infty} \int_{0}^{1} f_{n}(x) dx
\end{align*}
(4) 次の積分の値を求めよ.
\begin{align*}
\int_{0}^{1} \lim_{n \to \infty} f_{n}(x) dx
\end{align*}

\footnote[0]{
質問等はmino2357あっとgmail.comまで
}

\thispagestyle{empty}


\end{document}